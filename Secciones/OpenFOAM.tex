\chapter{OpenFOAM}

\section{Introducción} %% <--- Podríamos no poner nombre de "INTRODUCCIÓN" y directamente dejar ese texto debajo del capítulo para que denote que eso es como de lo que se va hablar en este capítulo, algo muy similar al libro de Moukalled, no cree?

\section{Requerimientos} %%% <-- En esta parte a qué se refiere con Requerimientos, no logro entender esa parte, a requerimientos de Hardware o Requerimientos conceptuales

\section{Pre Procesamiento}

%%% Considero que sería mejor dejar solo "Directorios" y dentro de eso dejar "system", "constant" y "0", después dejar otra que se llamen "Diccionarios de caso" y denrto de esa misma dejar todos los posibles diccionarios que ahí, tanto los básicos como "blockMeshDict", "controlDict", "fvSolution", "fvSchemes", etc. En resumen, esta primera propuesta tendría el siguiente esquema:

% ├─ Directorios
% │	  ├─ system
% │	  ├─ constant
% │	  └─ 0
% │
% └─ Diccionarios
% 	  ├─ blockMeshDict
% 	  ├─ controlDict
% 	  ├─ fvSchemes
% 	  └─ fvSolution

%%% Adicionalmente, podríamos dejar adicionalmente lo de condiciones de borde (hablando Neuman o Drichlet, que es con la que se configura las condiciones de borde), luego, ahí podríamos hablar de los esquemas de interpolación de phi en los bordes de la cara. Ampliando lo anterior a:

%
% ├─ Condiciones de borde
% │	  ├─ Neuman
% │	  ├─ Drichlet
% │	  └─ Aplicaciones en OpenFOAM
% │
% └─ Esquemas de interpolación
% 	  ├─ upwind
% 	  ├─ linearUpwind
% 	  ├─ limtedLinear
% 	  └─ interfaceCompression, etc.
%
%
%
	\subsection{Configuración de los directorios de caso}
	\subsection{System}
	\subsubsection{ControlDict}
	\subsubsection{Diccionarios de caso}
	\subsubsection{Medios Porosos}
	\subsubsection{FvOptions}
	\subsection{Constant}
	\subsection{Condiciones iniciales}
\section{Procesamiento}
	\subsection{Procesamiento en paralelo}
\section{Post Procesamiento}
	\subsection{Herramientas}
